\section{Costs and revenues}
In pursuit of our Smart Energy Devices project, managing costs and assessing potential revenues are critical aspects. This section outlines the financial considerations associated with the development and implementation of our BLDC sensored motor controller.

\subsection{Budget Allocation}
Our project operates within a budget of 100 euros per group, excluding the cost of a reasonably sized bare PCB. This allocation covers various expenses, including components, tools, and potential unforeseen costs during the project's lifecycle.

\subsubsection{Component Orders}
All component orders are made through approved suppliers listed on Brightspace, with Conrad or Farnell being highly recommended due to their advantageous shipping policies. To streamline the ordering process, groups are required to follow these steps:

\begin{enumerate}
    \item Create a comprehensive component list in MS Excel using the provided template on Brightspace.
    \item Rename the file as \textbf{PROJE4 2023 Component Order List Group Number.xlsx}, where \textbf{Number} corresponds to the group's assigned number in Brightspace.
    \item Double-check the suitability of the component packages for hand soldering.
    \item Verify component availability and delivery times.
    \item Submit the order list to Zoja Donné (\textbf{Z.Donne@hhs.nl}), with the course organizer (\textbf{Stephen O’Loughlin S.D.Oloughlin@hhs.nl}) in CC.
    \item Regularly check with Zoja regarding delivery times until components are received.
\end{enumerate}

\subsection{Expense Tracking}
To ensure transparency and accountability, each group must diligently track all purchased items and include a detailed record in the final report.

\subsection{Revenue Projection}
While our primary focus is on academic and technical achievements, the potential for future revenue generation should not be overlooked. The skills developed during this project, coupled with the innovative nature of our sensored BLDC motor controller, may open opportunities for collaboration, licensing, or even commercialization.

\subsection{Cost-Benefit Analysis}
A comprehensive cost-benefit analysis will be conducted at the project's conclusion, considering both tangible and intangible factors. This analysis will provide insights into the efficiency of resource utilization, the project's overall viability, and potential areas for improvement in future iterations.

\subsection{Financial Planning}
Effective financial planning is essential for successful execution of the project. Regular budget reviews, coupled with prudent decision making regarding expenditures, will contribute to the overall financial health of the project.

In conclusion, the judicious management of costs and a forward-looking perspective on potential revenues are integral components of our Smart Energy Devices project. This financial approach ensures the sustainability of the project, fosters responsible resource utilization, and sets the stage for future innovations in the realm of smart energy devices.
